\chapter{INTRODUCTION}

\pagebreak

\begin{center}
{\LARGE\textbf{INTRODUCTION}}
\end{center}


\section{Introduction}
Introduction \begin{math} P: \widehat{G} \rightarrow G \end{math} introduction introduction introduction introduction introduction introduction introduction introduction introduction introduction introduction introduction introduction introduction introduction introduction introduction introduction introduction introduction introduction introduction 


\section{History}
Introduction introduction introduction introduction introduction introduction introduction introduction introduction introduction introduction introduction introduction introduction introduction introduction introduction introduction introduction introduction introduction introduction introduction \cite{ref1}, \cite{ref2}.  introductionIntroduction introduction introduction introduction introduction introduction introduction introduction introduction introduction introduction introduction introduction introduction introduction introduction introduction introduction introduction introduction introduction introduction introduction introductionIntroduction introduction introduction introduction introduction introduction introduction introduction introduction introduction introduction introduction introduction introduction introduction introduction introduction introduction introduction introduction introduction introduction introduction introductionIntroduction introduction introduction introduction introduction introduction introduction introduction introduction introduction introduction introduction introduction introduction introduction introduction introduction \cite{ref5, ref6, ref7}. 
\begin{longtable}{|c|c|c|c|}
\caption{A simple longtable example}\\
\hline
\textbf{First entry} & \textbf{Second entry} & \textbf{Third entry} & \textbf{Fourth entry} \\
\hline
\endfirsthead
\multicolumn{4}{c}%
{\tablename\ \thetable\ -- \textit{Continued from previous page}} \\
\hline
\textbf{First entry} & \textbf{Second entry} & \textbf{Third entry} & \textbf{Fourth entry} \\
\hline
\endhead
\hline \multicolumn{4}{r}{\textit{Continued on next page}} \\
\endfoot
\hline
\endlastfoot
1 & 2 & 3 & 4 \\ 1 & 2 & 3 & 4 \\ 1 & 2 & 3 & 4 \\ 1 & 2 & 3 & 4 \\
1 & 2 & 3 & 4 \\ 1 & 2 & 3 & 4 \\ 1 & 2 & 3 & 4 \\ 1 & 2 & 3 & 4 \\
1 & 2 & 3 & 4 \\ 1 & 2 & 3 & 4 \\ 1 & 2 & 3 & 4 \\ 1 & 2 & 3 & 4 \\
1 & 2 & 3 & 4 \\ 1 & 2 & 3 & 4 \\ 1 & 2 & 3 & 4 \\ 1 & 2 & 3 & 4 \\
1 & 2 & 3 & 4 \\ 1 & 2 & 3 & 4 \\ 1 & 2 & 3 & 4 \\ 1 & 2 & 3 & 4 \\
1 & 2 & 3 & 4 \\ 1 & 2 & 3 & 4 \\ 1 & 2 & 3 & 4 \\ 1 & 2 & 3 & 4 \\
1 & 2 & 3 & 4 \\ 1 & 2 & 3 & 4 \\ 1 & 2 & 3 & 4 \\ 1 & 2 & 3 & 4 \\
1 & 2 & 3 & 4 \\ 1 & 2 & 3 & 4 \\ 1 & 2 & 3 & 4 \\ 1 & 2 & 3 & 4 \\
1 & 2 & 3 & 4 \\ 1 & 2 & 3 & 4 \\ 1 & 2 & 3 & 4 \\ 1 & 2 & 3 & 4 \\
1 & 2 & 3 & 4 \\ 1 & 2 & 3 & 4 \\ 1 & 2 & 3 & 4 \\ 1 & 2 & 3 & 4 \\
1 & 2 & 3 & 4 \\ 1 & 2 & 3 & 4 \\ 1 & 2 & 3 & 4 \\ 1 & 2 & 3 & 4 \\
1 & 2 & 3 & 4 \\ 1 & 2 & 3 & 4 \\ 1 & 2 & 3 & 4 \\ 1 & 2 & 3 & 4 \\
\end{longtable}
\section{Background}
introduction introduction introduction introduction introduction introduction 
introductionIntroduction introduction introduction introduction introduction introduction introduction introduction introduction introduction introduction introduction introduction introduction introduction introduction introduction introduction introduction introduction introduction introduction introduction introductionIntroduction introduction introduction introduction introduction introduction introduction introduction introduction introduction introduction introduction introduction introduction introduction introduction introduction introduction introduction introduction introduction introduction introduction introductionIntroduction introduction introduction introduction introduction introduction introduction introduction introduction introduction introduction introduction introduction introduction introduction introduction introduction introduction introduction introduction introduction introduction introduction introductionIntroduction introduction introduction introduction introduction introduction introduction introduction introduction introduction introduction introduction introduction introduction introduction introduction introduction introduction introduction introduction introduction introduction introduction introduction as in table \ref{tab12}.

\begin{sidewaystable}[!htbp]
\caption{Comparison between Wired and wireless networks}
\begin{center}

\begin{tabular}{p{.2\textwidth}p{.3\textwidth}p{.4\textwidth}}
\hline
Specifications       & Wired network    & Wireless network       \\ \hline
Speed of operation   & Higher           & lower compare to wired networks, But advanced wireless technologies such as LTE, LTE-A and WLAN-11ad will make it possible to achieve speed par equivalent to wired network \\
System Bandwidth     & High              & Low, as Frequency Spectrum is very scarse resource    \\
Cost                 & Less as cables are not expensive      & More as wireless subscriber stations, wireless routers, wireless access points and adapters are expensive                 \\
Installation         & Wired network installation is cumbersome and it requires more time   & Wireless network installation is easy and it requires less time             \\
Mobility            & Limited, as it operates in the area covered by connected systems with the wired network                & Not limited, as it operates in the entire wireless network coverage         \\
Transmission medium     & copper wires, optical fiber cables, ethernet   & EM waves or radiowaves or infrared  \\
Network coverage extension    & requires hubs and switches for network coverage limit extension  & More area is covered by wireless base stations which are connected to one another.    \\
Applications           & LAN (Ethernet), MAN        & WLAN, WPAN(Zigbee, bluetooth), Infrared, Cellular(GSM,CDMA, LTE) \\
Channel Interference and signal power loss & Interference is less as one wired network will not affect the other                          & Interference is higher due to obstacles between wireless transmitter and receiver e.g. weather conditions, reflection from walls, etc.                                      \\
QoS (Quality of Service)                   & Better           & Poor due to high value of jitter and delay in connection setup  \\
Reliability           & High compare to wireless counterpart, as manufactured cables have higher performance due to existence of wired technology since years. & Reasonably high, This is due to failure of router will affect the entire network.    \\ \hline
\end{tabular}

    \end{center}
\label{tab12}
\end{sidewaystable}

\subsection{History}
Introduction introduction introduction introduction introduction introduction introduction introduction introduction introduction introduction introduction introduction introduction introduction introduction introduction introduction\\ 
I like Operating Systems (\ac{os}), Artificial Intelligent (\ac{ai}), Tranmission Control Protocol (\ac{tcp}), Machine Learning (\ac{ml}), Convolutional Neural Network (\ac{cnn}), and Data Mining (\ac{dm}). 




\subsection{history2}
\ac{os} is one of my favourite subjects.
\ac{ai} is considered to be ,,,,introduction introduction introduction introduction introduction introduction introduction introduction introduction introduction introduction introduction introduction introduction introduction introduction introductionIntroduction introduction introduction introduction introduction introduction introduction introduction introduction introduction introduction introduction introduction introduction introduction introduction introduction introduction introduction introduction introduction introduction introduction introductionIntroduction introduction introduction introduction introduction introduction introduction introduction introduction introduction introduction introduction introduction introduction introduction introduction introduction introduction introduction introduction introduction introduction introduction introductionIntroduction introduction introduction introduction introduction introduction introduction introduction introduction introduction introduction introduction introduction introduction introduction introduction introduction introduction introduction introduction introduction introduction introduction introductionIntroduction introduction, as illustrated in Eq.\ref{eq20}.

\begin{equation}
Z=\sum x1+x2+x3
\label{eq20}
\end{equation}

introduction introduction introduction introduction introduction introduction introduction introduction introduction introduction introduction introduction introduction introduction introduction introduction introduction introduction introduction introduction introduction introductionIntroduction introduction introduction introduction introduction introduction introduction introduction introduction introduction introduction introduction introduction introduction introduction introduction introduction introduction introduction introduction introduction introduction introduction introductionIntroduction introduction introduction introduction introduction introduction introduction introduction introduction introduction introduction introduction introduction introduction introduction introduction introduction introduction introduction introduction introduction introduction introduction introduction

%for long equations
% \begin{equation}
% \begin{split}
% F = \{F_{x} \in  F_{c} &: (|S| > |C|) \\
%  &\quad \cap (\text{minPixels}  < |S| < \text{maxPixels}) \\
%  &\quad \cap (|S_{\text{conected}}| > |S| - \epsilon) \}
% \end{split}
% \end{equation}

% \begin{multline}
% F = \{F_{x} \in  F_{c} : (|S| > |C|) \cap 
% (minPixels  < |S| < maxPixels) \\ \cap 
% (|S_{conected}| > |S| - \epsilon)
%   \}
% \end{multline}

% \begin{equation}
% \begin{aligned}
% F ={} & \{F_{x} \in  F_{c} : (|S| > |C|) \\
%       & \cap (\mathrm{minPixels}  < |S| < \mathrm{maxPixels}) \\
%       & \cap (|S_{\mathrm{conected}}| > |S| - \epsilon)\}
% \end{aligned}
% \end{equation}



\section{Proposed}
{\em Introduction} introduction introduction introduction introduction introduction introduction introduction introduction introduction introduction introduction introduction introduction introduction introduction introduction introduction introduction introduction introduction introduction introduction introduction.\newline 


The chapters are:
\begin{itemize}
      \item \textbf{Chapter 1} in this chapter we will introduce
      \item \textbf{Chapter 2} in this chapter we will introduce
      \item \textbf{Chapter 3} in this chapter we will introduce
      \item \textbf{Chapter 4} in this chapter we will introduce
      
\end{itemize}